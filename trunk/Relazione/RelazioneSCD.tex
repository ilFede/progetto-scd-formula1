\documentclass[a4paper,10pt]{book}
\usepackage[utf8]{inputenc}

\begin{document}

  \chapter{Introduzione}
  
    Il progetto didattico per il corso di Sistemi Concorrenti e Distribuiti consiste nell'analisi e la risoluzione 
    delle problematiche di progettazione di un simulatore concorrente e distribuito 
    di una competizione paragonabile ad una gara Formula 1.

    Il sistema da simulare dovrà prevedere:
    \begin{itemize}
      \item Un circuito, possibilmente selezionabile in fase di configurazione, dotato almeno della pista e della corsia di 
        rifornimento.
        Entrambe dovranno essere soggette a regole congruenti di accesso, condivisione, 
        tempo di percorrenza, condizioni atmosferiche, ecc.
      \item Un insieme configurabile di concorrenti, ciascuno con caratteristiche specifiche di prestazione, risorse, 
        strategia di gara, ecc.
      \item Un sistema di controllo capace di riportare costantemente, consistentemente e separatamente, 
        lo stato della competizione, le migliori prestazioni (sul giro, per sezione di circuito) e anche la 
        situazione di ciascun concorrente rispetto a specifici parametri tecnici
      \item Una particolare competizione, con specifica configurabile della durata e controllo di terminazione 
        dei concorrenti a fine gara.
    \end{itemize}

\end{document}
