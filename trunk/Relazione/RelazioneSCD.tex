
\documentclass[a4paper,11pt, twoside]{book}

\usepackage[italian]{babel}
\usepackage[utf8x]{inputenc}
\usepackage{graphicx}
\usepackage{float}
\usepackage{url}
\usepackage[writefile]{listings}
\usepackage[Sonny]{fncychap}
\usepackage{fancyhdr}
\usepackage{supertabular}
\usepackage{longtable}
\usepackage[table]{xcolor}
\usepackage{array}
\usepackage{multirow}

\usepackage[italian]{babel}
\usepackage[utf8x]{inputenc}
\usepackage{graphicx}
\usepackage{float}

\usepackage{url}
\usepackage[writefile]{listings}
\usepackage{supertabular}
\usepackage{longtable}
\usepackage{booktabs}
\usepackage[table]{xcolor}


\usepackage{amssymb,amsmath}

\pagestyle{fancy}

%Impostazione capitoletti pagina
%\renewcommand{\chaptermark}[1]%
%{\markboth{\MakeUppercase{Capitolo \thechapter:\ #1}}{}}
%\renewcommand{\sectionmark}[1]%
%{\markright{\MakeUppercase{Sezione \thesection:\ #1}}}

%Impostazione linee di inizio e fine pagina
\renewcommand{\headrulewidth}{0.5pt}
\renewcommand{\footrulewidth}{0.2pt}

\newcommand{\helv}{%
\fontfamily{phv}\fontseries{b}\fontsize{9}{11}\selectfont}

\fancyhf{}
\fancyhead[LE,RO]{\helv \thepage}
\fancyhead[LO]{\helv \rightmark}
\fancyhead[RE]{\helv \leftmark}
\clearpage{\pagestyle{empty}\cleardoublepage}

\lstnewenvironment{xml}{\lstset{basicstyle=\footnotesize, frame=trbl, breaklines=true}}{} 
\begin{document}

  %Inizio Titolo
  \thispagestyle{empty}
    
  \begin{flushright}
    {Università degli Studi di Padova}\linebreak[1]
    \textbf{Corso di Laurea \linebreak Magistrale in Informatica} \linebreak \linebreak \linebreak \linebreak
  \end{flushright}
  
  \begin{center}
    {\texttt{Relazione del progetto per il corso di Sistemi Concorrenti e Distribuiti \linebreak \linebreak \linebreak \linebreak \linebreak}}
  \end{center}
  
  \begin{center}
    \texttt{\huge{\textbf{Simulatore concorrente e distribuito di una competizione di formula 1}}} \linebreak \linebreak \linebreak \linebreak \linebreak \linebreak \linebreak \linebreak \linebreak
  \end{center}
  
  
  \begin{flushleft}
    Studente: Cacco Federico\\Matricola: 624686\\Data: 16 giugno 2012
  \end{flushleft}

  \newpage
  \pagenumbering{Roman}
  \setcounter{page}{1}
  %Fine titolo

  \tableofcontents
  \newpage
  
  \chapter{Introduzione}
    \setcounter{page}{1}
    \pagenumbering{arabic}
    
    \section{Scopo del progetto}
      Il progetto didattico per il corso di Sistemi Concorrenti e Distribuiti consiste nell'analisi e la risoluzione 
      delle problematiche di progettazione di un simulatore concorrente e distribuito 
      di una competizione paragonabile ad una gara Formula 1.

      Il sistema da simulare dovrà prevedere:
      \begin{itemize}
	\item Un circuito, possibilmente selezionabile in fase di configurazione, dotato almeno della pista e della corsia di 
	  rifornimento.
	  Entrambe dovranno essere soggette a regole congruenti di accesso, condivisione, 
	  tempo di percorrenza, condizioni atmosferiche, ecc.
	\item Un insieme configurabile di concorrenti, ciascuno con caratteristiche specifiche di prestazione, risorse, 
	  strategia di gara, ecc.
	\item Un sistema di controllo capace di riportare costantemente, consistentemente e separatamente, 
	  lo stato della competizione, le migliori prestazioni (sul giro, per sezione di circuito) e anche la 
	  situazione di ciascun concorrente rispetto a specifici parametri tecnici
	\item Una particolare competizione, con specifica configurabile della durata e controllo di terminazione 
	  dei concorrenti a fine gara.
      \end{itemize}
    
    \section{Struttura del documento}
      ******** DO-TO ********
  
  \chapter{Analisi dei requisiti}
    I requisiti funzionali obbligatori del progetto sono elencati nella tabella \ref{tbl:RequisitiFunzionaliObbligatori}
    
    \begin{longtable}{|p{2cm}|p{8cm}|}
      \toprule
	\bfseries{CODICE} & \bfseries{DESCRIZIONE} \\\hline
      \endfirsthead
      RFOBB-01 & Presenza di un circuito nella competizione \\\hline
      RFOBB-02 & Presenza di piloti nella competizione \\\hline 
      RFOBB-03 & Presenza di un sistema di controllo della competizione \\\hline
      RFOBB-04 & Il circuito deve essere dotato di una pista e della corsia di rifornimento \\\hline
      RFOBB-05 & La pista deve essere soggetta a regole congruenti di accesso \\\hline
      RFOBB-06 & La corsia dei box deve essere soggetta a regole congruenti di accesso \\\hline
      RFOBB-07 & La pista e la corsia box devono essere condivisibili tra i piloti \\\hline
      RFOBB-08 & La pista e la corsia box devono avere tempi di percorrenza verosimili \\\hline
      RFOBB-09 & La pista e la corsia box devono essere soggette a condizioni atmosferiche \\\hline
      RFOBB-10 & I concorrenti devono possedere personali caratteristiche di prestazione  \\\hline
      RFOBB-11 & I concorrenti devono possedere una strategia di gara  \\\hline
      RFOBB-12 & I concorrenti devono possedere una vettura con specifiche caratteristiche prestazionali  \\\hline
      RFOBB-13 & Il sistema di controllo deve riportare lo stato della competizione  \\\hline
      RFOBB-14 & Il sistema di controllo deve riportare le prestazioni e o stato dei piloti  \\\hline
      RFOBB-15 & Il sistema di controllo deve tener traccia delle migliori prestazioni  \\\hline
      RFOBB-16 & Durata e condizione meteo della gara devono essere configurabili  \\\hline
      RFOBB-17 & Presenza di un controllo di terminazione dei concorrenti a fine gara  \\\hline
      \caption{Requisiti funzionali obbligatori}
      \label{tbl:RequisitiFunzionaliObbligatori}
    \end{longtable}

    I requisiti funzionali opzionali del progetto sono invece elencati nella tabella \ref{tbl:RequisitiFunzionaliOpzionali}
    
    \begin{longtable}{|p{2cm}|p{8cm}|}
      \toprule
	\bfseries{CODICE} & \bfseries{DESCRIZIONE} \\\hline
      \endfirsthead
      RFOPZ-01 & Il circuito può essere scelto in fase di configurazione \\\cline{1-2}
      RFOPZ-02 & I piloti devono poter essere configurabili \\\cline{1-2}
      \caption{Requisiti funzionali opzionali}
      \label{tbl:RequisitiFunzionaliOpzionali}
    \end{longtable}
    
  
  \chapter{Progettazione}
    \section{Entità coinvolte}
      Durante la fase di progettazione sono state individuate le seguenti entità costitutive:
      
      \begin{itemize}
        \item Circuito
        \item Piloti
        \item Controller
        \item Gara 
      \end{itemize}
      
      Di seguito verranno descritte in modo più approfondito per chiarire il loro compito all'interno del
      programma.
      
    \section{Descrizione delle entità coinvolte}
      \subsection{Circuito}
        L'entità circuito rappresenta il circuito vero e proprio nel quale si svolge la gara, dato che non deve 
        eseguire alcuna azione si
        tratta di una entità passiva che ha lo solo scopo di accogliere i piloti impegnati ad affrontare la
        competizione.
        Dovrà comunque regolamentare la loro attività garantendo che vengano rispettate delle regole di accesso congrue
        ad una ipotetica competizione reale. 
        Le caratteristiche e le regole che il circuito dovrà fornire saranno le seguenti:
        
        \begin{itemize}
          \item Dovrà essere specificata una lunghezza del circuito
          \item Il circuito deve avere una larghezza variabile durante la sua percorrenza, non in tutti i tratti sarà
                ad esempio possibile eseguire un sorpasso per via dello spazio limitato
          \item Nel circuito dovranno essere presenti sia tratti curvi che rettilinei, che avranno quindi diverse velocità
                di percorrenza
          \item Dovranno esserci dei punti per il rilevamento delle prestazioni
          \item Dovrà essere presente una corsia per i box, con relative regole di accesso
	  \item Tutte queste opzioni dovranno poter essere configurate mediante file di configurazione
        \end{itemize}
        
        Questa risorsa sarà condivisa da più piloti allo stesso tempo, sarà quindi necessario gestire
        tutte le problematiche di concorrenza derivate.
        
        
        
        
      
      \subsection{Piloti}
        I piloti sono le entità che dovranno percorrere il circuito per gareggiare nella competizione,
        la percorrenza del circuito dovrà avvenire seguendo le regole e i vincoli imposti dal circuito
        stesso.
        
        Dato che nelle competizioni reali ogni pilota ha le proprie caratteristiche, come ad esempio la prontezza di riflessi
        o la capacità di valutare i punti esatti dove effettuare le staccate \footnote{Con il termine staccata si indica 
        la fase in cui un pilota dopo un tratto rettilineo frena bruscamente per iniziare l'inserimento in curva},
        anche in questo caso ad ogni pilota dovranno essere associate delle caratteristiche, chiamate skill, che possano
        modellarne il comportamento in pista.
        I piloti avranno poi una loro personale strategia di gara, questo consentirà loro di effettuare
        i sorpassi non solo durante la percorrenza del giro, ma anche raggiungendo un buon compromesso
        tra numero di soste ai box e prestazioni su pista. Infatti maggiori saranno le soste e migliori saranno le prestazioni
        sul giro, dato che il pilota avrà presumibilmente un minore quantitativo di benzina, e quindi un minor peso, nella vettura.
        Bisognerà fare comunque attenzione che la programmazione delle soste sia tale da consentire al pilota di concludere la gara,
        evitando che esso finisca l carburante prima del suo termine
        
        Le caratteristiche che dovranno avere i piloti saranno quindi:
        
        \begin{itemize}
	  \item Nome 
	  \item Numero
	  \item Skill misurante la capacità di eseguire il prima possibile le accelerazioni in uscita dalle curve
	  \item Skill misurante la capacità di ritardare il più possibile la staccata prima di un inserimento in curva 
	        \footnote{La staccata dovrà essere comunque effettuata in modo tale da garantire la 
	        corretta percorrenza della curva}
	  \item Un strategia di gara che regoli le soste ai box
	\end{itemize}
	
        
        Ad ogni pilota sarà poi assegnata una vettura, anche questa dotata di determinate caratteristiche che influenzeranno il suo
        comportamento in pista. Esse saranno:
        
        \begin{itemize}
          \item Nome del costruttore
          \item Forza frenante
          \item Potenza di accelerazione
          \item Tenuta in curva
          \item Consumo per chilometro
          \item Velocità massima
          \item Livello attuale di carburante
        \end{itemize}

      \subsection{Controller}
        % \<TO DO\>
        Il controller di gara si occupa di visualizzare l'andamento della competizione, mostrando
        i tempi su giro di ogni pilota, i tempi intermedi di ogni pilota e il tempo del giro migliore.
        
      \subsection{Gara}
        % \<TO DO\>
        Questa entità si occupa di inizializzare l'intera competizione e di avviare la gara.
        Contiene meteo, numero di giri, piloti.....
          
    \section{Distribuzione}
        % \<TO DO\>
        
  \chapter{Implementazione}
    Dato che il progetto presenta alcune problematiche di concorrenza, per la sua realizzazione è stato scelto
    di utilizzare il linguaggio di programmazione ADA.
    
    Di seguito verrà descritto come sono state realizzate le varie entità
    
    \section{Circuito}
      
  
  \chapter{Compilazione ed esecuzione}

\end{document}
